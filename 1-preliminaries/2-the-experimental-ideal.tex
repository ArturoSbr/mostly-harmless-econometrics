% Imports
\documentclass[12pt]{article}
\usepackage[margin=1in]{geometry} 
\usepackage{
    amsmath, amsthm, amssymb
}

% Begin document
\setlength{\parindent}{0pt}
\begin{document}

% Title
\title{Mostly Harmless Econometrics Notes}
\author{Arturo Soberon}
\date{}
\maketitle

% Main
\setcounter{section}{1}
\section{The Experimental Ideal}
This chapter defines what a potential outcome is and shows how non-experimental
data may result in inaccurate estimates because of something called 
\textit{selection bias}.

Imagine we were facing the question \textit{Do hospitals have a positive
effect on people's health?} At first glance, one may be tempted to use survey
data and compare the health status of people who have been hospitalized in the
last 12 months and those who have not been hospitalized in the same period.

The NHIS survey grades the self-reported health status of responders from 1 to 
5 (the higher the better). Subtracting the mean average health of the
non-hospitalized group ($3.93$) from that of the hospitalized group ($3.21$)
results in an estimated value of $-0.72$, which suggests that being
hospitalized negatively impacts people's health. However, this result is
misleading because of selection bias.

Let $Y_i$ be the $i$-th person's health and $D_i$ their treatment status such
that

\begin{equation*}
    D_i = 
    \begin{cases}
        0, \text{$i$ is not treated} \\
        1, \text{$i$ is treated}
    \end{cases}
\end{equation*}

In the context of our example, $D_i = 1$ would mean that the $i$-th
person was hospitalized at least once in the last twelve months.

Any person is either hospitalized or not. Hence, we say that any person has two
potential health variables

\begin{equation*}
    \textit{Potential Outcome}_i = 
    \begin{cases}
        Y_{0i}, \text{$i$ was not hospitalized} \\
        Y_{1i}, \text{$i$ was hospitalized}
    \end{cases}
\end{equation*}

A potential outcome is basically what would have happend in a
parallel reality where the $i$-th person did the opposite of what we witnessed.
That is, if $i$ was hospitalized, then $Y_{0i}$ would be that person's observed
health in a world where they were not hospitalized. We would like to know the
difference between $Y_{1i}$ and $Y_{0i}$ because it is the causal effect of
going to the hospital for person $i$.

The \textit{observed} outcome can be expressed as

\begin{equation*}
    Y_i = Y_{0i} + (Y_{1i} - Y{0i}) D_i
\end{equation*}

With this definition, subtracting the average health of the non-hospitalized
group from the hospitalized group can be rewritten as

\begin{equation*}
    \begin{split}
        E(Y_i | D_i = 1) - E(Y_i | D_i = 0)
            & = \, E(Y_{0i} + (Y_{1i} - Y_{0i}) D_i | D_i = 1) \\
            & \qquad - E(Y_{0i} + (Y_{1i} - Y_{0i}) D_i | D_i = 0) \\
            & = \, E(Y_{0i} + (Y_{1i} - Y_{0i}) | D_i = 1)
                - E(Y_{0i} | D_i = 0) \\
            & = \, E(Y_{1i} - Y_{0i} | D_i = 1)
            + \big[ E(Y_{0i} | D_i = 1) - E(Y_{0i} | D_i = 0) \big]
    \end{split}
\end{equation*}

The term $E(Y_{1i} - Y_{0i} | D_i = 1)$ is the Average Treatment Effect on the
Treated (ATT) because it is the expected value of the health difference on the
population that was hospitalized. The second term
$E(Y_{0i} | D_i = 1) - E(Y_{0i} | D_i = 0)$ is the mean health of the
hospitalized group in a world where they were not hospitalized minus the mean
health of the non-hospitalized group in a world where they were not, in fact,
hospitalized. This term is known as \textit{selection bias}.

It is reasonable to think that people from the non-hospitalized group have
better health than people from the hospitalized group (because their health is
so good that they did not need to be hospitalized in the first place). Hence,
the term $E(Y_{0i} | D_i = 1)$ must be much smaller than $E(Y_{0i} | D_i = 0)$.
As a result, the selection bias term is negative and it outweighs the ATT,
which is why the observed difference in means is negative ($-0.72$).

Let us now imagine that we could randomly assign people to be hospitalized,
whether they needed it or not (i.e., imagine we could randomly assign people's
treatment status). This would imply that $D_i$ would now be random, which is
useful because the conditional terms in the equation above can now be ignored.
More specifically, if $D_i$ is randomly assigned, then
$E(Y_i | D_i = 0) = E(Y_i | D_i = 1) = E(Y_i)$. Therefore, the naive difference
in means turns into

\begin{equation*}
    E(Y_{1i} - Y_{0i} | D_i = 1)
        + \big[ E(Y_{0i} | D_i = 1) - E(Y_{0i} | D_i = 0) \big]
        = \, E(Y_{1i} - Y_{0i}) + \big[ E(Y_{0i}) - E(Y_{0i}) \big]
\end{equation*}

As we can see, the ATT is now the Average Treatment Effect (ATE) and the
selection bias term cancels out. In conclusion, randomly assigning the
treatment allows us to accurately estimate the causal effect of hospitals on
people's health by calculating a naive comparison of means. In contrast, the
same estimator yields an inaccurate estimate when using non-experimental data
because the treatment is not randomly assigned. This does not mean that
experimental data is flawless, but it can overcome selection bias when done
correctly.

Suppose the treatment effect is the same for everyone such that
$Y_{1i} - Y_{0i} = \rho \, \forall i$ . With constant treatment effects we can
express the observed outcome as

\begin{equation*}
    Y_i = \alpha + \rho D_i + \eta_i
\end{equation*}

Evaluating the conditional expected value of this variable with treatment
switched on ($D_i = 1$) and off ($D_i = 0$) results in

\begin{align*}
    & E(Y_i | D_i = 1) = \alpha + \rho + E(\eta_i | D_i = 1) \\
    & E(Y_i | D_i = 0) = \alpha + E(\eta_i | D_i = 0)
\end{align*}

The term $\eta_i$ is the stochastic part of the regression, which captures the
difference between our estimated value $E(Y_i | D_i)$ and $Y_i$. Expressing
$\eta$ as $Y_{0i} - E(Y_{0i})$ is useful because selection bias comes from
differences in $Y_{0i}$ between both groups.\footnote{
    Remember that selection bias equals
    $E(Y_{0i} | D_i = 0) - E(Y_{0i} | D_i = 1)$. In other words, it captures
    the difference in the expected outcomes of both groups in a world where
    neither one receives treatment.
}

The equivalent form of the naive difference in means using linear regression is

\begin{equation*}
    E(Y_i | D_i = 1) - E(Y_i | D_i = 0) = \rho + \big[
        E(\eta_i | D_i = 1) - E(\eta_i | D_i = 0) \big]
\end{equation*}

which implies that the term in brackets on the RHS represents selection bias.
When $\eta_i$ is correlated with $D_i$, the expected value of the error term is
different between both groups. Hence, selection bias amounts to correlation
between $\eta_i$ and $D_i$. If the two are uncorrelated, the term in brackets
cancels out.

% End document 
\end{document}
