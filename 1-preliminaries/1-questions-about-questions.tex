% Imports
\documentclass[12pt]{article}
\usepackage[margin=1in]{geometry} 
\usepackage{
    amsmath, amsthm, amssymb
}

% Begin document
\begin{document}

% Title
\title{Mostly Harmless Econometrics Notes}
\author{Arturo Soberon}
\date{}
\maketitle

% Main
\setcounter{section}{0}
\section{Questions About Questions}
This chapter presents four questions that researchers should ask themselves to
carry out a successful research project. The authors refer to these questions
as \textit{Frequently Asked Questions} (FAQs) throughout the book.

\subsection{What is the causal relationship of interest?}
Knowing the estimated causal effect of an intervention is useful in making
predictions about the consequences of changing circumstances or policies. We
may be interested in knowing the causal increment to wages an individual would
receive if he or she got an additional year of education.

\subsection{What is the ideal experiment to capture said causal effect?}
Ideal experiments are usually hypothetical because they often are too expensive
or unethical. Regardless, they are worth thinking about because if your causal
relationship of interest cannot be estimated in a world where anything goes,
then it is unlikely that you will be able to find good estimates on a tight
budget or with non-experimental data. Additionally, contemplating the ideal
experiment helps you highlight the factors that you would like to manipulate
(treatment) or hold constant (controls).

Questions that cannot be answered by an ideal experiment are called
Fundamentally Unidentified Questions (FUQs). Such questions are ones where
the treatment cannot be randomly assigned by an omnipotent scientist, such as
race.

\subsection{What is your identification strategy?}
This whole book focuses on using non-experimental data to estimate causal
effects. In this context, an \textit{identification strategy} is the manner
in which the researcher uses data that was not generated by a randomized trial
to approximate a randomized experiment.

\subsection{What is your mode of statistical inference?}
Describe the population studied, the sample used and the assumptions made when
constructing your standard errors.

% End document 
\end{document}
